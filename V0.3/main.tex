% ----------------------------------------------------------------
\documentclass[zihao=-4,linespread=1.375,fontset=windows]{ctexbook}
% ----------------------------------------------------------------
% ----------------------------------------------------------------
% 页面设置
\usepackage{geometry}
\geometry{a4paper,left=2.5cm,right=2.5cm,top=2.5cm,bottom=2.5cm}
% ----------------------------------------------------------------
% 字体设置
% 将\songti,\heiti,\kaishu设置为可加粗可倾斜的字体
\xeCJKsetup{
AutoFakeBold={1.75},
AutoFakeSlant={true},
}
\newCJKfontfamily[Xsong]\Xsongti[AutoFakeBold,AutoFakeSlant]{SimSun}
\newCJKfontfamily[Xhei]\Xheiti[AutoFakeBold,AutoFakeSlant]{SimHei}
\newCJKfontfamily[Xkai]\Xkaishu[AutoFakeBold,AutoFakeSlant]{KaiTi}
\renewcommand{\songti}{\Xsongti}
\renewcommand{\heiti}{\Xheiti}
\renewcommand{\kaishu}{\Xkaishu}
% ----------------------------------------------------------------
% 页眉页脚设置
\usepackage{fancyhdr}
\pagestyle{fancy}
\fancyhf{}
\fancyfoot[LE,RO]{\thepage}
\renewcommand{\headrulewidth}{0pt}
\renewcommand{\footrulewidth}{0pt}
% 使\cleardoublepage建立的空白页无页眉页脚
\makeatletter
\def\cleardoublepage{
  \clearpage
  \if@twoside
    \ifodd
      \c@page
    \else
      \begingroup
        \mbox{}
        \thispagestyle{empty}
        \newpage
        \if@twocolumn\mbox{}\newpage\fi
      \endgroup
    \fi
  \fi}
\makeatother
% ----------------------------------------------------------------
% 章节标题样式设置
\ctexset{
  chapter = {
    beforeskip = {0\ccwd},
    afterskip = {1.75\ccwd},
    format = \centering\zihao{3}\heiti,
    pagestyle = fancy,
  },
  section = {
    beforeskip = {0.5\ccwd},
    afterskip = {0.5\ccwd},
    format = \indent\zihao{4}\heiti,
  },
  subsection = {
    beforeskip = {0\ccwd},
    afterskip = {0\ccwd},
    format = \indent\bfseries\zihao{-4}\songti,
  },
}
% ----------------------------------------------------------------
% 目录设置
\ctexset{
  contentsname = 目~~~录,
}
\usepackage{titletoc}
% 参考!
\titlecontents{chapter}
	[0\ccwd]
	{\addvspace{0.5\ccwd}\zihao{-4}\songti}
	{\thecontentslabel~~}
	{}
	{\titlerule*[0.5pc]{$\cdot$}\contentspage}
\titlecontents{section}
	[1.75\ccwd]
	{\addvspace{0.5\ccwd}\zihao{-4}\songti}
	{\thecontentslabel~~~}
	{}
	{\titlerule*[0.5pc]{$\cdot$}\contentspage}
\titlecontents{subsection}
	[3\ccwd]
	{\addvspace{0.5\ccwd}\zihao{-4}\songti}
	{\thecontentslabel~~~}
	{}
	{\titlerule*[0.5pc]{$\cdot$}\contentspage}
% ----------------------------------------------------------------

% ----------------------------------------------------------------
% 数学宏集
\usepackage{mathtools}
\usepackage{amsmath}
\usepackage{amsfonts}
\usepackage{amssymb}
\usepackage{amsthm}
% ----------------------------------------------------------------
% 定理环境
\newtheorem{thm}{定理}[chapter]
\newtheorem{cor}{推论}[chapter]
\newtheorem{lem}{引理}[chapter]
\newtheorem{prop}{命题}[chapter]
\theoremstyle{definition}
\newtheorem{defn}{定义}[chapter]
\newtheorem{question}{问题}[chapter]
\theoremstyle{remark}
\newtheorem{rem}{注记}[chapter]
\newtheorem{example}{例}[chapter]
\numberwithin{equation}{chapter}
% ----------------------------------------------------------------
% 数学符号
% 1.
\newcommand{\norm}[1]{\left\Vert#1\right\Vert}
\newcommand{\abs}[1]{\left\vert#1\right\vert}
\newcommand{\set}[1]{\left\{#1\right\}}
% 2.
\newcommand{\N}{\mathbb N}
\newcommand{\Z}{\mathbb Z}
\newcommand{\Q}{\mathbb Q}
\newcommand{\R}{\mathbb R}
\newcommand{\C}{\mathbb C}
% ----------------------------------------------------------------
% 常用宏包
\usepackage{comment}
\usepackage{tabularx} % tabularx表格环境
\usepackage{array} % 控制表格的列
\usepackage{booktabs} % 三线表样式
\usepackage{graphicx} % 插入图片
% ----------------------------------------------------------------
% 参考文献
\usepackage[numbers,sort&compress]{natbib}
% ----------------------------------------------------------------
% 超链接
\usepackage{hyperref} % 为减少可能的冲突,习惯上将hyperref宏包放在其它宏包之后调用。
%\hypersetup{
%  colorlinks=true,
%  linkcolor=blue,
%  filecolor=magenta,
%  urlcolor=cyan,
%}
\hypersetup{hidelinks} % 取消链接的颜色和边框
% ----------------------------------------------------------------
\begin{document}
% ----------------------------------------------------------------
% ----------------------------------------------------------------
% 数学宏集
\usepackage{mathtools}
\usepackage{amsmath}
\usepackage{amsfonts}
\usepackage{amssymb}
\usepackage{amsthm}
% ----------------------------------------------------------------
% 定理环境
\newtheorem{thm}{定理}[chapter]
\newtheorem{cor}{推论}[chapter]
\newtheorem{lem}{引理}[chapter]
\newtheorem{prop}{命题}[chapter]
\theoremstyle{definition}
\newtheorem{defn}{定义}[chapter]
\newtheorem{question}{问题}[chapter]
\theoremstyle{remark}
\newtheorem{rem}{注记}[chapter]
\newtheorem{example}{例}[chapter]
\numberwithin{equation}{chapter}
% ----------------------------------------------------------------
% 数学符号
% 1.
\newcommand{\norm}[1]{\left\Vert#1\right\Vert}
\newcommand{\abs}[1]{\left\vert#1\right\vert}
\newcommand{\set}[1]{\left\{#1\right\}}
% 2.
\newcommand{\N}{\mathbb N}
\newcommand{\Z}{\mathbb Z}
\newcommand{\Q}{\mathbb Q}
\newcommand{\R}{\mathbb R}
\newcommand{\C}{\mathbb C}
% ----------------------------------------------------------------

% ----------------------------------------------------------------
\ZTFLH{XXXX} % 中图分类号
\MJ{公开} % 密级
\UDC{XXXX} % UDC
\XXDM{10094} % 学校代码
\XW{硕士学位论文}
\LX{学术学位}
\Title{中文题名} % 中文题名
\Etitle{英文题名} % 英文题名
\Name{XXX} % 研究生姓名
\Mentor{XXX{\hspace{1\ccwd}}职称} % 指导教师
\Major{数学} % 学科专业
\Direction{XXXX} % 研究方向
\ProposalDate{XXXX年XX月XX日} % 论文开题日期
\Date{二〇二五年XX月XX日} % 日期
% ----------------------------------------------------------------
 % 填写论文信息
\begin{titlepage}
\begin{center}

\vspace*{0cm}

\begin{table}[h]
\centering
\begin{tabularx}{\linewidth}{
>{\bfseries\zihao{-3}\heiti\raggedleft\arraybackslash}p{7\ccwd}
>{\bfseries\zihao{-3}\heiti\centering\arraybackslash}p{7.5\ccwd}
X
>{\bfseries\zihao{-3}\heiti\raggedleft\arraybackslash}p{7\ccwd}
>{\bfseries\zihao{-3}\heiti\centering\arraybackslash}p{7.5\ccwd}
}
  学校代码: & \textbf{\XXDMInnerValue} &  & 密级: & \MJInnerValue \\ \cline{2-2} \cline{5-5}
\end{tabularx}

\vspace{1.5\ccwd}

\begin{tabularx}{\linewidth}{
>{\bfseries\zihao{-3}\heiti\raggedleft\arraybackslash}p{7\ccwd}
>{\bfseries\zihao{-3}\heiti\centering\arraybackslash}p{7.5\ccwd}
X
>{\bfseries\zihao{-3}\heiti\raggedleft\arraybackslash}p{7\ccwd}
>{\bfseries\zihao{-3}\heiti\centering\arraybackslash}p{7.5\ccwd}
}
  中图分类号: & \textbf{\ZTFLHInnerValue} &  & \textbf{UDC}: & \textbf{\UDCInnerValue} \\ \cline{2-2} \cline{5-5}
\end{tabularx}
\end{table}

\vspace{3\ccwd}

\includegraphics[scale=0.3]{settings/HebeiNormalUniversity}

\vspace{1.5\ccwd}

{\bfseries\zihao{-0}\songti \XWInnerValue}

\vspace{1\ccwd}

{\bfseries\zihao{-2}\songti (\LXInnerValue)}

\vspace{4\ccwd}

{\bfseries\zihao{2}\heiti \TitleInnerValue}

\vspace{1\ccwd}

{\zihao{3} \textbf{\EtitleInnerValue}}

\vfill

\begin{table}[h]
\centering
\begin{tabular}{>{\bfseries\zihao{4}\songti}l>{\bfseries\zihao{4}\songti}l}
\makebox[8\ccwd][s]{研究生姓名:} & \NameInnerValue \\[1\ccwd]
\makebox[8\ccwd][s]{指导教师:} & \MentorInnerValue \\[1\ccwd]
\makebox[8\ccwd][s]{学科专业:} & \MajorInnerValue \\[1\ccwd]
\makebox[8\ccwd][s]{研究方向:} & \DirectionInnerValue \\[1\ccwd]
\makebox[8\ccwd][s]{论文开题日期:} & \ProposalDateInnerValue
\end{tabular}
\end{table}

\vspace{2.5\ccwd}

{\bfseries\zihao{-3}\heiti \DateInnerValue}
%\vspace{1.5\ccwd}

\vspace*{0.5cm}

\end{center}
\end{titlepage}

\newpage
\thispagestyle{empty}
 % 制作封面
\begin{titlepage}
\begin{center}

\vspace*{-0.525cm}

\begin{table}[h]
\centering
\begin{tabularx}{\linewidth}{
>{\zihao{-3}\heiti\raggedleft\arraybackslash}p{7\ccwd}
>{\zihao{-3}\heiti\centering\arraybackslash}p{7.5\ccwd}
X
>{\zihao{-3}\heiti\raggedleft\arraybackslash}p{7\ccwd}
>{\zihao{-3}\heiti\centering\arraybackslash}p{7.5\ccwd}
}
  学校代码: & \textbf{\XXDMInnerValue} &  & 密级: & \MJInnerValue \\ \cline{2-2} \cline{5-5}
\end{tabularx}

\vspace{0.8\ccwd}

\begin{tabularx}{\linewidth}{
>{\zihao{-3}\heiti\raggedleft\arraybackslash}p{7\ccwd}
>{\zihao{-3}\heiti\centering\arraybackslash}p{7.5\ccwd}
X
>{\zihao{-3}\heiti\raggedleft\arraybackslash}p{7\ccwd}
>{\zihao{-3}\heiti\centering\arraybackslash}p{7.5\ccwd}
}
  中图分类号: & \textbf{\ZTFLHInnerValue} &  & \textbf{UDC}: & \textbf{\UDCInnerValue} \\ \cline{2-2} \cline{5-5}
\end{tabularx}
\end{table}

\vspace{3.5\ccwd}

\includegraphics[scale=0.3]{settings/HebeiNormalUniversity}

\vspace{1.75\ccwd}

{\bfseries\zihao{-0}\songti \XWInnerValue}

\vspace{0.8\ccwd}

{\zihao{-2}\songti (\LXInnerValue)}

\vspace{3.5\ccwd}

{\bfseries\zihao{2}\heiti \TitleInnerValue}

\vspace{0.75\ccwd}

{\zihao{3} \textbf{\EtitleInnerValue}}

\vfill

\begin{table}[h]
\centering
\begin{tabular}{>{\zihao{4}\songti}l>{\zihao{4}\songti}l}
\makebox[8\ccwd][s]{研究生姓名:} & \NameInnerValue \\[0.65\ccwd]
\makebox[8\ccwd][s]{指导教师:} & \MentorInnerValue \\[0.65\ccwd]
\makebox[8\ccwd][s]{学科专业:} & \MajorInnerValue \\[0.65\ccwd]
\makebox[8\ccwd][s]{研究方向:} & \DirectionInnerValue \\[0.65\ccwd]
\makebox[8\ccwd][s]{论文开题日期:} & \ProposalDateInnerValue
\end{tabular}
\end{table}

\vspace{2.5\ccwd}

%{\zihao{-3}\heiti \DateInnerValue}
\vspace{1.875\ccwd}

\vspace*{0.575cm}

\end{center}
\end{titlepage}

\newpage
\thispagestyle{empty}
 % 制作标题页
\begin{titlepage}

\begin{center}

  \vspace*{0.5\ccwd}

  {\bfseries\zihao{3}\heiti 学位论文原创性声明}

  \vspace{0.5\ccwd}

\end{center}

\vspace{0.5\ccwd}

本人所提交的学位论文《\TitleInnerValue》,是在导师的指导下,独立进行研究工作所取得的原创性成果。除文中已经注明引用的内容外,本论文不包含任何其他个人或集体已经发表或撰写过的研究成果。对本文的研究做出重要贡献的个人和集体,均已在文中标明。

本声明的法律后果由本人承担。

\vspace{1.5\ccwd}

\begin{table}[h]
  \raggedleft
  \begin{tabular}{rp{6\ccwd}rp{6\ccwd}}
  论文作者(签名): &  & 指导教师确认(签名): &  \\[0.25\ccwd]
  ~~~~~年~~~~~月~~~~~日 &  & ~~~~~年~~~~~月~~~~~日 & 
  \end{tabular}
\end{table}

\vspace{8.25\ccwd}

\begin{center}
  \vspace{0.5\ccwd}

  {\bfseries\zihao{3}\heiti 学位论文版权使用授权书}

  \vspace{0.5\ccwd}
\end{center}

\vspace{0.5\ccwd}

本学位论文作者完全了解河北师范大学有权保留并向国家有关部门或机构送交学位论文的复印件和磁盘,允许论文被查阅和借阅。本人授权河北师范大学可以将学位论文的全部或部分内容编入有关数据库进行检索,可以采用影印、缩印或其它复制手段保存、汇编学位论文。

(保密的学位论文在\underline{\hspace{3.5\ccwd}}年解密后适用本授权书)

\vspace{1.5\ccwd}

\begin{table}[h]
  \raggedleft
  \begin{tabular}{rp{6\ccwd}rp{6\ccwd}}
  论文作者(签名): &  & 指导教师(签名): &  \\[0.25\ccwd]
  ~~~~~年~~~~~月~~~~~日 &  & ~~~~~年~~~~~月~~~~~日 & 
  \end{tabular}
\end{table}

\end{titlepage}

\newpage
\thispagestyle{empty}
 % 制作学位论文原创性声明页
% ----------------------------------------------------------------
\frontmatter % 前言部分,页码使用小写罗马数字;其后的 \chapter 不编号。
\pagenumbering{Roman} % 页码使用大写罗马数字

\chapter*{摘\hspace{1.75\ccwd}要}
\addcontentsline{toc}{chapter}{中文摘要}
%\setcounter{page}{3}

XXXXXX

\vspace{1.25\ccwd}\noindent\textbf{关键词:} 关键词

\chapter*{\textbf{Abstract}}
\addcontentsline{toc}{chapter}{英文摘要}

XXXXXX

\vspace{1.25\ccwd}\noindent\textbf{Key Words:} Key Words


\tableofcontents

\chapter*{符号说明}
\addcontentsline{toc}{chapter}{符号说明}

\begin{table}[h]
	\centering
	\begin{tabular}{ll}
		\toprule
		符号 & 说明 \\
        \midrule
	    $\mathcal{L}(H_1,H_2)$ & Hilbert空间$H_1$到$H_2$的所有有界线性算子的集合 \\
        \bottomrule
	\end{tabular}
\end{table}

% ----------------------------------------------------------------
\mainmatter % 正文部分,页码使用阿拉伯数字,从 1 开始计数;其后的章节编号正常。

\chapter*{引\hspace{1.75\ccwd}言}
\addcontentsline{toc}{chapter}{引言}

\section{研究背景}

\cite{Cowen1978}

\section{研究内容}



\section{主要结果}




\chapter{章标题}

\section{节标题}

\subsection{小节标题}

国内外有些学者围绕培养制度、培养条件以及导师和学生的素质等方面讨论了研究生学位论文质量的影响因素。有学者认为,影响学位论文质量的原因是多方面的,其中最主要的影响因素是开展学位论文所必需的科研条件、学位论文所属学科的水平、学生以及导师的素质。

\subsection{小节标题}

\section{节标题}

\subsection{小节标题}


\chapter{B}

\section{b}

\subsection{(b)}

% ----------------------------------------------------------------
\backmatter % 后记部分,页码格式不变,继续正常计数;其后的\chapter 不编号。

\chapter*{结\hspace{1.75\ccwd}论}
%\addcontentsline{toc}{chapter}{结论}

结论


% 参考文献
\begin{thebibliography}{99}
\addcontentsline{toc}{chapter}{参考文献}

\bibitem{Cowen1978}
Cowen M~J, Douglas R~G.
\newblock Complex geometry and operator theory\allowbreak[J].
\newblock Acta Math., 1978, 141\allowbreak (3-4): 187-261.

\end{thebibliography}

%\bibliographystyle{settings/gbt7714-numerical_change}
%\bibliography{subdocuments/references}

\chapter{附\hspace{1.75\ccwd}录}

附录


\chapter*{后~~~记}
\addcontentsline{toc}{chapter}{后记}

后记内容


\chapter*{攻读学位期间取得的科研成果清单 }
\addcontentsline{toc}{chapter}{攻读学位期间取得的科研成果清单 }

XXXXXX

% ----------------------------------------------------------------
\end{document}
% ----------------------------------------------------------------
